\documentclass[conference]{IEEEtran}

\usepackage{booktabs}
\usepackage{graphicx}

\title{ECG Heartbeat Classification}
\author{Pham Dinh Duy}
\date{January 2026}

\begin{document}

\maketitle

\section{Introduction}
This work studies automated ECG heartbeat classification using deep learning.
A one-dimensional convolutional neural network is trained to categorize heartbeats into five clinically relevant classes.

\section{Dataset}
The ECG Heartbeat Categorization dataset is derived from the PhysioNet MIT-BIH Arrhythmia and PTB Diagnostic ECG databases.
Each sample represents a single segmented heartbeat of fixed length (188 samples).
MIT-BIH data are used for multi-class classification, while PTB data are used for normal versus abnormal waveform comparison.

\subsection{Class distribution (MIT-BIH train)}
The training data are highly imbalanced, with normal beats has the most counts.

\begin{center}
\begin{tabular}{l r}
\toprule
Class & Count \\
\midrule
N (0) & 72471 \\
S (1) & 2223 \\
V (2) & 5788 \\
F (3) & 641 \\
Q (4) & 6431 \\
\bottomrule
\end{tabular}
\end{center}

\section{Model}
Lightweight 1D CNN is used, consisting of two convolutional blocks with batch normalization, ReLU activation, and max pooling.
The extracted features are flattened and passed through a fully connected classifier.
Class imbalance is handled using weighted cross-entropy loss.

\section{Training Setup}
The model is trained for 15 epochs using the Adam optimizer with a learning rate of $10^{-3}$ and batch size 256.
Training and evaluation are performed on the MIT-BIH train and test splits.

\section{Results}

The CNN model was trained for 15 epochs on the MIT-BIH dataset using class-weighted cross-entropy loss.
Training accuracy increased steadily, while validation accuracy stabilized after approximately 8 epochs.

The final test accuracy achieved was \textbf{95.31\%}.

\subsection{Training}
Training and validation loss decreased gradually, suggesting effective optimization.
There were some small fluctuations in validation loss in later epochs, but no overfitting occurred.
Accuracy converges after mid-training epochs.

\subsection{Classification Performance}
Table-based metrics show strong performance on the dominant normal class (N) and high recall for class Q.
Precision for minority arrhythmia classes is lower, reflecting the strong class imbalance in the dataset.

\begin{itemize}
    \item Normal beats (N) achieved high precision and recall.
    \item Lesser count classes show less precision due to less sample counts.
\end{itemize}






\end{document}
